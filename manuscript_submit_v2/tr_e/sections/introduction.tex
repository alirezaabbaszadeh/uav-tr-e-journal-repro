Unmanned aerial vehicles (UAVs) are increasingly considered for time-sensitive urban pickup-and-delivery operations, either as standalone fleets or as complements to ground vehicles in the last mile \citep{murray2015,poikonen2017,otto2018}. Their appeal is clear: UAVs can bypass ground congestion, provide rapid response, and serve hard-to-reach locations. At the same time, real deployments remain constrained by endurance, payload limits, and safety requirements, which place routing and scheduling decisions at the center of operational feasibility \citep{dorling2017,tothvigo2014}.

From an operations-research perspective, time-window compliance is a defining service-level requirement. The VRP with time windows (VRPTW) and pickup-and-delivery with time windows (PDPTW) have long served as canonical models for such settings \citep{solomon1987,desrochers1992,parragh2008,ropke2006}. However, UAV logistics introduces an additional layer of operational risk that is often treated implicitly or ignored: communication reliability. Command-and-control and telemetry links depend on air-to-ground propagation and base-station geometry, leading to coverage holes and outage corridors even in dense urban areas \citep{alhourani2014,zeng2016,mozaffari2016}.

Communication outages are not merely a networking detail. They translate into operational unreliability: conservative contingency behaviors, safety-driven rerouting, mission aborts, and degraded effective fleet capacity. Yet most routing models optimize travel time or distance under deterministic constraints, while communications research typically optimizes network-centric objectives (coverage, rate, placement) rather than logistics KPIs. This disconnect motivates an integrated view in which communication reliability enters the routing objective as an operationally interpretable risk signal.

This paper targets Transportation Research Part E by treating communication outage probability as an operational attribute that interacts with service-level constraints. We incorporate an arc-level outage-risk model into a multi-UAV pickup-and-delivery routing problem and adopt soft time windows to explicitly trade off lateness against energy and reliability. Importantly, we complement the methodological contribution with an audited, reviewer-oriented evidence protocol: all reported numbers and artifacts are generated from a single locked campaign with frozen instances, manifests, and machine-checked claim guarding.

\subsection*{Contributions}
We make four contributions:
\begin{itemize}[leftmargin=*]
\item \textbf{Reliability-aware objective:} We integrate an arc-level communication outage risk with an energy proxy and soft time-window tardiness in a single calibratable objective, enabling explicit energy--risk--service tradeoffs.
\item \textbf{Operational risk modeling:} Using a standard urban air-to-ground link abstraction, we estimate per-arc outage risk by sampling along flight segments and associating each sample point to the best base station by SNR \citep{alhourani2014,zeng2016}.
\item \textbf{Solver stack with conservative inference:} We combine an OR-Tools heuristic for scalable solution construction, a HiGHS MIP model for certificate-gated exactness and bound reporting, and a hard time-window baseline (PyVRP). We enforce a size-dependent claim regime (exact-with-certificate, bound-gap, scalability-only) to prevent invalid inference \citep{highs,ortools,pyvrp}.
\item \textbf{Journal-grade reproducibility:} We provide a complete reviewer-ready package with manifests, frozen benchmarks, generated tables/figures, and anonymous/camera-ready bundles, following reproducible research best practices \citep{peng2011,stodden2016}.
\end{itemize}

The remainder of the paper reviews related work (Section~\ref{sec:related_work}), defines the problem and assumptions (Section~\ref{sec:problem}), presents the formulation and risk model (Sections~\ref{sec:formulation}--\ref{sec:risk_model}), describes the solution approach (Section~\ref{sec:methods}), and reports results and managerial insights (Sections~\ref{sec:results}--\ref{sec:insights}).
