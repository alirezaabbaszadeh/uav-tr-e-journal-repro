Unmanned aerial vehicles (UAVs) are increasingly considered for time-sensitive urban logistics, either as standalone fleets or as complements to ground distribution in the last mile \citep{murray2015,poikonen2017,otto2018}. Their operational promise is attractive for practitioners: bypassing road congestion, improving responsiveness, and serving locations that are costly for ground vehicles. At the same time, practical deployments remain constrained by limited endurance, payload capacity, and safety requirements, which makes routing and scheduling decisions central to feasibility and service quality \citep{dorling2017,tothvigo2014}.

From an operations-research perspective, service-level agreements are often encoded via time windows. The vehicle routing problem with time windows (VRPTW) and its extensions have therefore become canonical models for time-sensitive distribution and have motivated decades of algorithmic development \citep{solomon1987,desrochers1992,laporte2009}. In real operations, strict (hard) time windows can be brittle: as demand scales or uncertainty increases, feasibility may collapse and the operational meaning of a ``good'' solution becomes unclear. Soft time windows offer a policy alternative by allowing controlled lateness while explicitly pricing tardiness in the objective \citep{tothvigo2014,cordeau2002}.

UAV operations introduce an additional reliability dimension that is often treated implicitly or outside the routing model: communication connectivity. Command-and-control and telemetry links depend on air-to-ground propagation, environment geometry, and base-station availability. Even in dense urban areas, these factors can create outage corridors and coverage holes \citep{alhourani2014,zeng2016,mozaffari2016}. In practice, communication outages translate into operational unreliability: conservative contingency behaviors, mission aborts, rerouting, and degraded effective fleet capacity. For a TR-E audience, the key point is that communication reliability is not merely a networking detail; it directly affects logistics performance and risk exposure.

Existing research tends to separate these concerns. The VRP literature focuses on logistics KPIs under operational constraints, while communications literature typically optimizes network-centric objectives such as coverage, placement, or rate. This motivates an integrated view in which communication outage risk enters the routing objective as an operationally interpretable signal that can be calibrated against energy and service outcomes.

This paper targets Transportation Research Part E by framing communication risk as an operational reliability attribute and by pairing that model with a conservative, reviewer-oriented evidence protocol. We study a multi-UAV routing problem with mixed pickup and delivery demands, capacity propagation, and time-window SLAs. We incorporate an arc-level outage-risk estimate derived from an air-to-ground channel abstraction and a base-station layout, and we adopt soft time windows to preserve feasibility under stress while exposing the cost of lateness.

\subsection*{Contributions}
We make four contributions:
\begin{itemize}[leftmargin=*]
\item \textbf{Reliability-aware objective:} We integrate arc-level communication outage risk with an energy proxy and soft time-window tardiness in a single calibratable objective, enabling explicit energy--risk--service tradeoffs.
\item \textbf{Operational risk modeling:} Using a standard urban air-to-ground link abstraction, we estimate per-arc outage risk by sampling along flight segments and associating each sample point to the best base station by SNR \citep{alhourani2014,zeng2016}.
\item \textbf{Solver stack with conservative inference:} We combine an OR-Tools heuristic for scalable planning, a HiGHS MIP model for certificate-gated exactness and bounds, and a hard time-window baseline (PyVRP). We enforce a size-dependent claim regime (exact-with-certificate, bound-gap, scalability-only) to prevent invalid inference \citep{highs,ortools,pyvrp}.
\item \textbf{Journal-grade reproducibility:} All reported numbers and assets are generated from a single immutable evidence campaign with manifests, audit gates, and machine-checked claim guarding, consistent with reproducible research recommendations \citep{peng2011,stodden2016}.
\end{itemize}

The remainder of the paper reviews related work (Section~\ref{sec:related_work}), defines the problem and assumptions (Section~\ref{sec:problem}), presents the formulation and risk model (Sections~\ref{sec:formulation}--\ref{sec:risk_model}), describes the solution approach (Section~\ref{sec:methods}), and reports results and managerial insights (Sections~\ref{sec:results}--\ref{sec:insights}).
