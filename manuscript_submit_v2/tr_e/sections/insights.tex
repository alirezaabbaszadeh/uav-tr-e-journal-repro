\label{sec:insights}

Transportation Research Part E emphasizes actionable, operations-facing implications. Using the robustness slices in the locked campaign, we highlight three managerial insights that follow directly from the audited sensitivity analyses.

\subsection{Base-station density is an operational lever for reliability}
Communication risk is strongly shaped by base-station density $B$. Table~\ref{tab:managerial_support} and Figure~\ref{fig:bs_delta_effect} show that densification can sharply reduce mean arc outage risk across a range of time-window tightness settings. For example, at $\Delta=10$ minutes, OR-Tools mean arc outage risk drops from approximately 0.247 at $B=4$ to 0.0768 at $B=7$ and 0.0366 at $B=10$\evid{TAB_MANAGERIAL_SUPPORT_method_ortools_main_B_4_Delta_min_10_risk_mean}\evid{TAB_MANAGERIAL_SUPPORT_method_ortools_main_B_7_Delta_min_10_risk_mean}\evid{TAB_MANAGERIAL_SUPPORT_method_ortools_main_B_10_Delta_min_10_risk_mean}. This suggests that network planning and infrastructure decisions can be as consequential as algorithm selection when reliability targets are binding.

\input{generated/tables/tab_managerial_support.tex}

\begin{figure}[t]
\centering
\includegraphics[width=0.92\linewidth]{generated/figures/fig_bs_delta_effect.pdf}
\caption{Base-station density and time-window tightness jointly shape arc outage risk.}
\label{fig:bs_delta_effect}
\end{figure}

\subsection{Hard time windows can overstate service quality by excluding failure cases}
Hard time windows yield excellent service KPIs \emph{when feasible} (near-100\% on-time and near-zero tardiness). However, the feasibility results in Section~\ref{sec:results} show that hard feasibility can collapse as instances scale and/or time windows tighten. Operationally, this means that strict service metrics may be optimistic if they are implicitly conditioned on feasibility. Soft time windows provide a controllable alternative: they preserve feasibility while exposing the cost of lateness in a transparent, calibratable way.

\subsection{Penalty calibration is a policy decision}
The soft time-window penalty weight $\lambda_{TW}$ and the communication risk weight $\lambda_{out}$ should be interpreted as policy levers. Increasing $\lambda_{TW}$ shifts solutions toward improved time-window compliance at the expense of energy (and potentially altered risk exposure), as illustrated in Figure~\ref{fig:lambda_tw_tradeoff}. In practice, this supports a structured approach to SLA design: choose penalty weights to meet target on-time service levels, quantify the cost of compliance, and use infrastructure levers (e.g., base-station densification) when the risk constraint is the binding driver.

\begin{figure}[t]
\centering
\includegraphics[width=0.92\linewidth]{generated/figures/fig_tradeoff_lambda_tw.pdf}
\caption{Soft time-window penalty weight induces an energy--service tradeoff.}
\label{fig:lambda_tw_tradeoff}
\end{figure}
