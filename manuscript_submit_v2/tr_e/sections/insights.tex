\label{sec:insights}

Transportation Research Part E emphasizes actionable operational insights. Using the robustness campaign, we highlight three managerial takeaways.

\subsection{Base-station density is an operational lever for reliability}
Communication risk is strongly shaped by base-station density $B$. For example, at $\Delta=10$ minutes in the robustness slices, OR-Tools mean arc outage risk drops from approximately 0.247 at $B=4$ to 0.0768 at $B=7$ and 0.0366 at $B=10$\evid{TAB_MANAGERIAL_SUPPORT_method_ortools_main_B_4_Delta_min_10_risk_mean}\evid{TAB_MANAGERIAL_SUPPORT_method_ortools_main_B_7_Delta_min_10_risk_mean}\evid{TAB_MANAGERIAL_SUPPORT_method_ortools_main_B_10_Delta_min_10_risk_mean}. This suggests that service-area infrastructure and network planning can be as impactful as algorithm choice when reliability targets are binding.

\subsection{Hard time windows can overstate service quality by excluding failure cases}
When hard time windows are feasible, they deliver excellent service KPIs (near-100\% on-time and near-zero tardiness). However, at larger sizes ($N=40$) feasibility can collapse (Tables~\ref{tab:feas_A}--\ref{tab:feas_B}), meaning that reported service KPIs may be conditioned on feasibility and therefore optimistic. Soft time windows provide a controllable alternative: they preserve feasibility and make the energy--service tradeoff explicit.

\subsection{Penalty calibration is a policy decision}
The soft time-window penalty weight $\lambda_{TW}$ (and risk weight $\lambda_{out}$) should be interpreted as a policy knob. Figure~\ref{fig:lambda_tw_tradeoff} shows that increasing $\lambda_{TW}$ shifts solutions toward improved service compliance at the cost of higher energy (and potentially different risk exposure). In practice, this allows operators to tune routes to meet SLA targets while quantifying the cost of compliance.
