\label{sec:limitations}

Our study is designed to provide a reproducible, operations-facing view of reliability-aware UAV routing under time-window pressure. Several limitations should be kept in mind when interpreting the results.

\paragraph{Simplified energy model.}
We use a linear distance-based energy proxy. Real UAV power consumption depends on payload, flight dynamics, wind, speed profiles, and battery characteristics \citep{dorling2017}. A richer energy model would improve absolute realism but would complicate large-scale reproducible experimentation.

\paragraph{Simplified radio model.}
The communication-risk signal is derived from a static air-to-ground pathloss abstraction and does not model network traffic, interference, handover delays, or congestion. The intent is comparative and sensitivity analysis, not prediction of absolute outage rates in a specific deployed cellular network.

\paragraph{Capacity and pickup/delivery abstraction.}
We model mixed pickup and delivery quantities without paired pickup--delivery precedence constraints between customer nodes. Pickups are implicitly returned to the depot at route end. This abstraction is appropriate for many last-mile collection/delivery tasks but does not cover full PDPTW-style job pairing.

\paragraph{Conservative claim regime.}
To avoid invalid inference, we make the following non-claims explicit:
\begin{itemize}[leftmargin=*]
\item We claim \emph{exactness} only when HiGHS provides an optimality certificate on $N\le 10$.
\item We report bounds and gaps only for $N\in\{20,40\}$ when both incumbent and bound are finite and compatible.
\item We do not report or interpret bounds/gaps for $N=80$; these results are scalability-only.
\end{itemize}

Future work can incorporate richer energy and network models, multi-objective tradeoff frontiers, and stochastic service times, while retaining the evidence-locking and audit mechanisms that support reviewer-grade reproducibility.
