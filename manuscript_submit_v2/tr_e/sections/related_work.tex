\label{sec:related_work}

\subsection{Vehicle routing with time windows and soft service constraints}
The vehicle routing problem (VRP) and its time-window variants are foundational models for time-sensitive distribution. Early formulations trace back to \citet{dantzig1959}, while benchmark-driven VRPTW research was catalyzed by the instances of \citet{solomon1987}. Exact methods based on column generation and branch-and-price have provided strong solution approaches and have shaped the modern methodological toolbox \citep{desrochers1992}. Broader perspectives and surveys are available in \citet{laporte2009,tothvigo2014}.

Modern VRP performance is often driven by large-scale neighborhood search and metaheuristics. Representative frameworks include tabu search \citep{savelsbergh1985,gendreau1994}, adaptive large neighborhood search (ALNS) \citep{shaw1998,pisinger2007}, constraint-based neighborhood operators \citep{rousseau2002}, high-performance local search hybrids \citep{braysy2004}, and surveys of dynamic routing settings \citep{pillac2013}. Extensions with pickup and delivery quantities further increase complexity and require careful load propagation modeling \citep{parragh2008,ropke2006}.

Learning-augmented approaches for combinatorial optimization are also emerging \citep{cappart2023}, but our emphasis here is on conservative, reviewer-auditable OR baselines and operational sensitivity analysis.

Hard time windows impose strict feasibility, which can be brittle when demand scales or when service conditions are stressed. Soft time windows (STW) are therefore widely used when operators prefer explicit cost-of-lateness tradeoffs over strict feasibility \citep{taillard1997soft,cordeau2002,tothvigo2014}. More general flexible-window cost models have also been studied \citep{sungur2006flex}. Our work adopts STW to preserve operational feasibility while enabling calibrated energy--risk--service tradeoffs that are interpretable for decision makers.

\subsection{UAV logistics routing and energy-aware planning}
UAV logistics has produced a broad family of routing and scheduling models, spanning pure-drone delivery, hybrid truck-and-drone systems, and multi-UAV fleet coordination \citep{murray2015,agatz2018,poikonen2017,mathew2015}. Surveys summarize civil UAV optimization models and emphasize endurance, safety, and energy constraints as primary drivers \citep{otto2018}. Energy consumption is especially critical for multirotor UAVs with limited endurance and payload-dependent efficiency \citep{dorling2017}.

Rather than proposing a new flight-dynamics or battery model, we focus on an operational reliability factor that is often missing from routing-centric formulations: communication outages. We therefore use a distance-based energy proxy that is sufficient to expose the central tradeoffs while keeping the experimental protocol reproducible at campaign scale.

\subsection{Communication-aware UAV operations and outage risk}
Air-to-ground communication performance depends strongly on geometry, environment type, and the probability of line-of-sight (LoS) conditions. A common modeling approach combines a logistic LoS probability with excess losses for LoS and non-LoS propagation, yielding a tractable pathloss abstraction \citep{alhourani2014,zeng2016}. Surveys and tutorials provide broader perspectives on UAV communications, channel modeling, and cellular integration \citep{khawaja2019,fotouhi2019,mozaffari2019tutorial,zeng2019proc,zeng2019mwc,hayat2020survey}. Standards-oriented work also highlights the practical relevance of aerial connectivity for cellular networks \citep{3gpp36777}.

Communication-aware routing for logistics has received comparatively less attention than either VRP with operational constraints or network-centric UAV optimization. We bridge this gap by mapping link-level outage events to an arc-level operational risk metric and integrating it directly into route planning. The goal is not to introduce a new physical-layer model, but to provide an interpretable reliability signal that supports sensitivity analysis over base-station density, time-window tightness, and risk-sampling resolution.
