\label{sec:related_work}

\subsection{Vehicle routing with time windows and pickup-and-delivery}
The vehicle routing problem (VRP) and its time-window variant (VRPTW) are foundational models for time-sensitive distribution. Early formulations trace back to \citet{dantzig1959}, while benchmark-driven VRPTW algorithm development was catalyzed by the instances of \citet{solomon1987} and subsequent exact approaches based on column generation and branch-and-price \citep{desrochers1992}. Comprehensive overviews and modern perspectives can be found in \citet{laporte2009,tothvigo2014}.

Pickup-and-delivery problems introduce coupled pickup and delivery quantities, precedence relationships, and more complex load propagation, substantially increasing the combinatorial difficulty \citep{parragh2008,ropke2006}. High-performing heuristic frameworks include tabu search, adaptive large neighborhood search (ALNS), and unified multi-attribute VRP frameworks \citep{savelsbergh1985,gendreau1994,ropke2006,vidal2014}.

Hard time windows impose strict feasibility, which can be brittle in real operations. Soft time windows (STW) instead allow controlled violations via penalty terms and have long been used when strict compliance is unrealistic or when operators want an explicit cost of tardiness tradeoffs \citep{cordeau2002,tothvigo2014}. Our work adopts STW to maintain operational feasibility under stress and to enable meaningful energy--risk--service calibration.

\subsection{UAV logistics routing and energy-aware planning}
UAV logistics has produced a rich family of routing and scheduling models, including pure-drone delivery and hybrid truck-and-drone systems \citep{murray2015,agatz2018,poikonen2017}. Surveys summarize civil UAV optimization models and highlight energy and safety constraints as primary drivers \citep{otto2018}. Energy consumption is especially critical for multirotor UAVs with limited endurance and payload-dependent flight efficiency \citep{dorling2017}. Multi-robot and multi-UAV package delivery planning has also been studied from a robotics perspective \citep{mathew2015}.

In contrast to developing new energy or flight-dynamics models, our focus is on the operational consequence of communication reliability. We therefore use a distance-based energy proxy sufficient to expose tradeoffs and to keep the model compatible with reproducible, campaign-scaled experimentation.

\subsection{Communication-aware UAV operations and outage risk}
Air-to-ground communication performance depends strongly on geometry and the probability of line-of-sight (LoS) conditions. A common modeling approach combines a logistic LoS probability with different excess losses for LoS and non-LoS (NLoS) propagation, yielding a tractable pathloss abstraction \citep{alhourani2014,zeng2016}. Such models have supported UAV altitude optimization, placement, and wireless coverage analysis \citep{mozaffari2016}.

Communication-aware routing for logistics has received comparatively less attention than either VRP with operational constraints or purely network-centric UAV optimization. We bridge this gap by mapping link-level outage events to an arc-level operational risk metric and integrating it directly into route planning. The goal is not a new physical-layer model, but an interpretable risk signal that enables sensitivity analysis over base-station density, time-window tightness, and risk-sampling resolution.
