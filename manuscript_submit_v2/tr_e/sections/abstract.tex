Urban UAV logistics promises fast, flexible last-mile pickup-and-delivery, but operational plans must satisfy delivery time windows while maintaining reliable command-and-control connectivity. We study a reliability-aware multi-UAV pickup-and-delivery problem that augments a classical VRPTW/PDPTW setting with (i) an arc-level communication outage risk estimated from an air-to-ground channel model and a base-station layout, and (ii) soft time windows that penalize tardiness rather than enforcing strict feasibility.

We evaluate a solver stack combining an OR-Tools heuristic for scalable planning, a HiGHS mixed-integer model for certificate-gated exact solutions on small instances and bounds on medium instances, and a hard time-window metaheuristic baseline (PyVRP). Using an audited, frozen benchmark suite with paired statistical testing, we quantify how time-window stress affects service quality and how reliability objectives trade off against energy and lateness. In the stressed family at $N=20$, the OR-Tools on-time rate decreases from 46.3\% to 36.3\% and mean total tardiness increases from 77.3 to 94.6 minutes\evid{C2_A_N20_on_time}\evid{C2_B_N20_on_time}\evid{C2_A_N20_tardiness}\evid{C2_B_N20_tardiness}. At $N=40$, soft time windows preserve feasibility (feasible rate $\approx 0.97$) while the hard-TW baseline returns no feasible solutions\evid{C3_A_N40_feasible_rate_a_n40_ortools}\evid{C3_B_N40_feasible_rate_b_n40_ortools}\evid{C3_A_N40_feasible_rate_a_n40_pyvrp}\evid{C3_B_N40_feasible_rate_b_n40_pyvrp}.

Finally, managerial sensitivity analyses show that base-station density and penalty calibration can materially shift the risk--service tradeoff, providing actionable levers beyond algorithm selection. All results are fully reproducible via a campaign-locked release that includes manifests, frozen instances, generated tables/figures, and anonymous/camera-ready bundles.
